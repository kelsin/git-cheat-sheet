\documentclass[letterpaper,12pt]{article}
\usepackage[cm]{fullpage}

\input xy
\xyoption{all}

\title{Git Cheat Sheet}
\author{Christopher Giroir}

\begin{document}

\maketitle

\section{Basics}

All commands in git can be called with \texttt{git command} or
\texttt{git-command}. If you want man page information, you need to use
\texttt{git-command} form.

\begin{verbatim}
# Both of the commands below work
git-diff
git diff

# Man pages are only in hyphen format
man git-diff
\end{verbatim}

\subsection{Creation}

\begin{description}
\item [git init] Making a brand new repo

\begin{verbatim}
# cd to the root directory of what you want to store
cd savingstar-awesome
git init
\end{verbatim}

\item [git clone] Making from another repo (for example, git hub)

\begin{verbatim}
git clone git@github.com:Kelsin/configs.git
# The "configs" folder is created by default
\end{verbatim}

\end{description}

\subsection{Commands simular to CVS/SVN}

The following commands behave very simular to cvs. Most of them support the
\texttt{--color} flag to make the output appear nice on an ansi-terminal and
also automatically page (so you don't need a \texttt{| less} when using them).

\begin{description}
\item [git status] Show what branch you're on and current changes (in index and out)
\item [git diff] Show working changes, changes between commits, etc.
\item [git blame] A better named cvs annotate
\item [git log] History of commits of a file, or repo.
\end{description}

\subsection{Commands that differ from CVS/SVN}
These commands either are not available in other version control system or
behave differently.

\begin{description}
\item [git commit] Commits to the current branch. Supply -v to get a full diff
  in the comments of your editor so you can see exactly what is being
  committed. This command only commits items from the index to the local repo
  (see the next section for more information).
\item [git branch] Deal with branches (more about this below)
\item [git stash] Saves your working directory away, can be recalled
  anytime. Handy when you aren't ready to even do a small branch commit and have
  to switch branches to fix a bug.
\item [gitk] Depending on how you installed git this program may be
  available. Shows you graphically what your commit tree/current branch looks
  like. On the mac we prefer GitX: http://gitx.laullon.com/
\end{description}

\subsection{Commits}
Each commit in Git stores who made the commit, a message, the date of the
commit, what commit is the parent of this commit (in a merge situation you can
have multiple parents) and a link to a whole file tree. Commits are not storing
diffs like in some other VCS's. In git each commit respresent the entire file
tree at that point in time. Behind the scenes git is clearly doing a lot of
clever stuff in order for this to be possible.

Each commit is labeled with a SHA-1 hash that takes all of this information into
account. If you need to change anything about a commit (the message, the author,
etc) it will change the SHA-1. It's cool to note that git DOES allow you to
write history in many ways. We'll look at a couple later in this document.

\subsection{Setting up your name and email}

To make sure your name appears correctly in github run these commands on acs-dev
(saves it in your home directory so this will now work on all vm's).

\begin{verbatim}
git config --global user.name "Christopher Giroir"
git config --global user.email "cgiroir@savingstar.com"
\end{verbatim}

\section{Repos}

There are four places that git stores data in a normal checkout.

\begin{description}
\item [Working Directory] These are the actual files you are editing in your
  project. It is possible to make a git repo without a working directory (for
  example to use as a shared server).
\item [Index] This is a local cache of files that are ready to be commited. You
  need to \texttt{add} files to this index in order to commit them. If you modify
  files after you add them, only the added version gets commited, not the new
  changes.
\item [Local Repository] This is the data inside the \texttt{.git} folder where
  git stores it's information. This will contain the entire history of the
  repository that you cloned from.
\item [Remote Repository] You can add remote repositories to your git
  repo. Unlike CVS and SVN you don't need to have a remote repository. In our
  use case these are most likely going to be repositories that live in
  gitorious.
\end{description}

\subsection{Simple Diagram of Git Commands}

This is very simplified but it shows the basic track of data during a normal
workflow.

\xymatrix{
  *++[F-,]{\txt{Working}} \ar@/^2pc/[r]^{\txt{git add}}&
  *++[F-,]{\txt{Index}} \ar@/^2pc/[r]^{\txt{git commit}}&
  *++[F-,]{\txt{Local Repo}} \ar@/^2pc/[r]^{\txt{git push}} \ar@/^2pc/[ll]^{\txt{git checkout}}&
  *++[F-,]{\txt{Remote Repo}} \ar@/^2pc/[l]^{\txt{git fetch}} \ar `d[dd] `[ddlll]^{\txt{git pull}} [lll] \\
  & & & \\
  & & &
}

\section{Branches}

Branches are extremely light weight in git. Git stores everything as a big tree
of commits and branches are just pointers to certain commits. In the following
diagram the top branch probably has a label (maybe \texttt{master}) on commit
\texttt{d} while the lower branch has a label (maybe \texttt{cs\_panel\_2009}) on
commit \texttt{f}.

\xymatrix{
  *++[o][F-]{a} \ar[r] &
  *++[o][F-]{b} \ar[r] \ar[rd] &
  *++[o][F-]{c} \ar[r] &
  *++[o][F-]{d} \\
  & & *++[o][F-]{e} \ar[r] &
  *++[o][F-]{f} \\
}

\subsection{Creation of a new branch}

\begin{verbatim}
# To create a brand new branch you can use
git branch <name>

# and then switch to it using
git checkout <name>

# You can also both of the above commands with
git checkout -b <name>

# To view branches
git branch

# To view all branches (including remote links)
git branch -a

# To delete a branch
git branch -d <name>

# If deleting this branch means that some commits will be unreachable
# (by branch labels) you might need to force it:
git branch -D <name>
\end{verbatim}

\subsection{Merging}

There are two types of merges that git can handle. The first is simular to
merges in CVS and Subversion and is called ``merge''. The other is called
``rebase'' and actually changes the SHA-1 hashes of the branch commits.

\subsubsection{Merges}
Using the diagram above a merge of the bottom branch to the top would yield this
result:

\xymatrix{
  *++[o][F-]{a} \ar[r] &
  *++[o][F-]{b} \ar[r] \ar[rd] &
  *++[o][F-]{c} \ar[r] &
  *++[o][F-]{d} \ar[r] &
  *++[o][F-]{g} \\
  & & *++[o][F-]{e} \ar[r] &
  *++[o][F-]{f} \ar[ur] \\
}

We now have a new commit (\texttt{g}) that contains the merge of the two
branches. When merging in this way you should make sure that your index is clean
(\texttt{git diff --cached HEAD} returns nothing).

When a clean merge is able to happen (no conflicts) you end up with both branch
pointers pointing to the new commit (\texttt{g}). At this point one of the two
branches can be deleted with \texttt{git branch -d} since all of the commits are
reachable from the other branch label.

\subsubsection{Merge Example}

\begin{verbatim}
# We want to merge coupon_fix into master
# First checkout master
git checkout master

# Now do the merge
git merge coupon_fix
\end{verbatim}

\subsubsection{Rebase}

A rebase will actually change the commits of the branch it's run on. This means
that this isn't a safe operation to run if the branch you are rebasing is being
shared and pushed around. This is a great thing to do when you are working on a
small branch (alone) and you want to commit it to a shared development
branch. It helps you keep the commit history very clean.

A rebase results in your changes being ``played back'' over the other
branch. This would result in the following graph:

\xymatrix{
  *++[o][F-]{a} \ar[r] &
  *++[o][F-]{b} \ar[r] &
  *++[o][F-]{c} \ar[r] &
  *++[o][F-]{d} \ar[dl] \\
  & & *++[o][F-]{e'} \ar[r] &
  *++[o][F-]{f'} \\
}

Right after the rebase command the upper branch label is still on commit
\texttt{d} while the lower branch label is on \texttt{f'}. At this point if you
run a merge on the master branch (as we did above in the example) you would get
a ``fast-forward'' merge since there is no computation that needs to be
done. The label for master would be moved from \texttt{d} to \texttt{f'}.

The end result is a cleaner history for the master branch. For complicated
branches this can't be done, but it can help prevent lots of tiny branches in
our commit history which can be hard to follow.

\subsubsection{Rebase Example}

\begin{verbatim}
# We want to merge coupon_fix into master using a rebase
# First checkout coupon_fix
git checkout coupon_fix

# Do the rebase
git rebase master

# Now checkout master and merge to fast-forward our branch label
git checkout master
git merge coupon_fix

# The checkout and rebase can be done as one command as well
# resulting in the following:
git rebase master coupon_fix
git checkout master
git merge coupon_fix
\end{verbatim}

\subsection{Sample Work Flow}
Here is a simple example of fixing a bug while working on another branch.

\begin{verbatim}
# New Project Time! Let's create a branch
> git checkout -b coupon_fix

# Do work
> git status
> git add <file> <file>
> git commit
# Repeat this a bunch

# Bug comes in! Make sure everything is commited on the coupon_fix branch
> git checkout master
# Edit files
> git commit -a
# Update live servers
local> git push
servers> git pull

# Back to new stuff!
> git checkout coupon_fix
\end{verbatim}

\end{document}
